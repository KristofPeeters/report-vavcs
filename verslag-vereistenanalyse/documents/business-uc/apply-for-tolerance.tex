\section{Apply for tolerance}

\begin{description}
	\item[Business Event Description] \ 
		\par After the deliberation of the last exam period (August - September) the
		student decides to tolerate a certain course. %TODO kan toch ook na Juni, mss
		% Kerst ?
	\item[Business Use Case Name] \ 
		\par Apply for tolerance
	\item[Triggering business event] \ 
		\par The student didn't pass for a certain course, even after the last
		examination period (August - September), and decides to tolerate it.
	\item[Preconditions] \
	\begin{itemize}
		\item The student obtained a score of 8 or 9 as highest score on an
		examination of a certain course (which he wants to tolerate), even after the
		last examination period (August - September).
	\end{itemize}
	\item[Active stakeholders] \ 
	\begin{itemize}
		\item Student: wants to obtain a tolerance for a certain course.
	\end{itemize}
	\item[Interested stakeholders] \ 
		\begin{itemize}
		\item Tolerance committee: checks if the student can indeed tolerate the
		course and will be notified when the student sends the form.
		\end{itemize}
	\item[Normal business flow] \ 
	\begin{enumerate}
	  	% 1
	  	\item The student takes the form for tolerating a course.
	  	% 2
	  	\item The student provides the necessary identifying information (found on
	  	his student card).
	  	% 3
	  	\item The student assigns for which course he wants to ask for a toleration.
	  	% 4
	  	\item The student sends the form to the tolerance committee.
	  	% 5
	  	\item The tolerance committee receives and checks the form and approves the
	  	tolerance, \textbf{include} \emph{(UC9: Approve tolerance)}.
	\end{enumerate}
	\item[Alternative business flow] \ 
		\par None
	\item[Exception business flow] \ 
		\par None
	\item[Outcome (postcondition)] \ 
		\par The student has completed the course he wanted to tolerate but he lost
		the study points of the tolerated course.
\end{description}
