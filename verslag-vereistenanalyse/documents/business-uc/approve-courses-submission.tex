\section{Approve courses submission}
\begin{description}
	\item[Business Event Description] \ 
		\par The educational committee receives an application for chosen courses from
		a student and needs to check if the student can really take these courses.
	\item[Business Use Case Name] \ 
		\par Approve Courses Submission
	\item[Triggering business event] \ 
		\par The educational committee receives an application for chosen courses.
	\item[Preconditions] \
	\begin{itemize}
		\item The educational committee has received the prefered selection of courses
		from the student.
	\end{itemize}
	\item[Active stakeholders] \ 
	\begin{itemize}
		\item Educational committee: checks if the selection of chosen courses
		received from the student is valid.
	\end{itemize}
	\item[Interested stakeholders] \ 
		\begin{itemize}
		\item University board: interested in new students, contract determines the
		inscription fee.
		\item Student: wants to see his chosen courses selection approved.
		\end{itemize}
	\item[Normal business flow] \ 
	\begin{enumerate}
	  	% 1
	  	\item The educational committee sees that the study contract of the student
	  	is a diploma contract.
	  	% 2
	  	\item The educational committee checks that if the student is a first year
	  	student he has chosen all the courses of the first phase of that program or
	  	that he has possible exemptions (e.g. from another education at the
	  	Wellington university or another university.)
	  	% 3
	  	\item The educational committee checks that the student meets all the entry
	  	requirements for the chosen courses.
	  	% 4
	  	\item The educational committee checks that if the student is a full-time
	  	student, the total number of taken study points ranges from 40 to 75. 
	  	% 5
	  	\item The educational committee concludes all the checks come out positive.
	\end{enumerate}
	\item[Alternative business flow] \
		\begin{description}
		\item[1a] The educational committee sees that the study contract of the
		student is a credit contract. 
			\begin{enumerate}
			  \item Resume the normal flow at step 3.
			\end{enumerate}
		\item[1b] The educational committee sees that the study contract of the
		student is an exam contract. 
			\begin{enumerate}
			  \item Resume the normal flow at step 3.
			\end{enumerate}
		\item[1c] The educational committee sees that the study contract of the
		student is a combination of various study contracts.
			\begin{enumerate}
			  \item The educational committee checks that the student doesn't try to
			  follow several credit contracts or exam contracts for courses under the
			  supervision of the same faculty.
			  \item The educational committee checks that the student didn't chose
			  an unfinished course that is included in different contracts at the same
			  time.
			  \item Resume the basic flow at step 3.
			\end{enumerate}
		\end{description}
		\item[4c] In case the student is a part-time student, the committee checkts
		that the total number of study points ranges from 0 to 30.
	\item[Exception business flow] \ 
	\begin{description}
		\item[5a] if one or more of the checks comes out negative then,
		\begin{enumerate}
		  \item The student is notified of the rejection of his selection.
		\end{enumerate}
	\end{description}
	\item[Outcome (postcondition)] \ 
		\par The selection of chosen courses is approved.
\end{description}
