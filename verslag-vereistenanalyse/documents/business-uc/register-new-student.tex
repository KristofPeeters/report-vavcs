\section{Register new student}
\label{register-new-student}

\begin{description}
	\item[Business Event Description] \
	\par A student wants to register at the Wellington
	University
	\item[Business Use Case Name] \
	\par Register new student
	\item[Triggering business event] \
	\par A student wants to register at the Wellington
	University. He is completely new and hasn't studied at Wellington before. He
	goes to the student administration building of Wellington University to make
	his inscription.
	\item[Preconditions] \ 
	\begin{itemize}
		\item The tudent hasn't yet studied at Wellington University.
		\item The student has a high school degree.
	\end{itemize}
	\item[Active stakeholders] \ 
	\begin{itemize}
	  	\item Student: wants to study at Wellington University.
		\item Registration committee: needs to complete the inscription of the student.
	\end{itemize}
	\item[Interested stakeholders] \ 
		\begin{itemize}
		  \item Board of the university: interested in new students.
		\end{itemize}
	\item[Normal business flow] \
	\begin{enumerate}
	  	% 1
	  	\item The student goes to the student administration building of Wellington
	  	University.
	  	% 2
	  	\item The student hands over the required information at the registration
	  	desk: home address, residence address, phone numbers, birth date, passport
	  	picture, register number, diplomas, if necessary evidence of matriculation 
	  	(if necessary)
	  	% 3
	  	\item The registration committee checks if the provided information is enough
	  	and is correct.
	  	% 4
	  	\item The registration committee makes the right documents: invoice,
	  	inscription attest, password for his student login, \ldots
	  	% 5
	  	\item The registration committee prints the student card with a unique
	  	student number and the passport picture of the student on it.
	  	% 6
	  	\item The registration committee grabs a few other documents like some
	  	university folders and a free buss card.
	  	% 7
	  	\item The registration committee hands all these documents and student card
	  	over to the student.
	  	% 8
	  	\item The student accepts all these documents and is now successfully
	  	inscribed to Wellington University.
	\end{enumerate}
	\item[Alternative business flow] \ 
	\begin{description}
  		\item[3a] The registration committee notices that some documents are missing
  		because the student has forgotten to bring them.
  		\begin{enumerate}
  			\item The registration committee makes the student aware that he/she
  			can't be inscribed without these documents.
  			\item The student leaves to retrieve the missing documents.
  			\item The student returns at the student administration to restart his/her
  			inscription.
  			\item Go to normal flow 4.
		\end{enumerate}
	\end{description}
	\item[Exception business flow] \
	\begin{description}
		\item[3a] after having checked the information provided by the student, the
		registration committee notices that some documents are missing. This time not
		because the student has forgotten them, but because the student just doesn't
		have these documents (high school diploma, evidence of having passed a
		matriculation, \ldots).
		\begin{enumerate}
		  \item The registration committee makes the student aware that he/she can't be
		  inscribed without these documents.
		  \item The student leaves and can't be inscribed until he obtains the missing
		  documents.
		\end{enumerate}
	\end{description}
	\item[Outcome (postcondition)] \
	\par The student is now fully inscribed at Wellington University and has
	received all the according documents, folders, buss card and student card.
\end{description}
