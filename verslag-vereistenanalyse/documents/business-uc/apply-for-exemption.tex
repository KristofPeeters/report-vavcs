\section{Apply for exemption}

\begin{description}
	\item[Business Event Description] \ 
		\par The student intends to register for at least one study contract in the
		upcoming acadmic year. In one of those contracts, there is a course he/she
		thinks he/she can substitute it with a course he/she followed in another
		contract. This substitution course was not necessarily taught at the
		university of Wellington.
	\item[Business Use Case Name] \ 
		\par Apply for exemption
	\item[Triggering business event] \ 
		\par The student thinks he can acquire an exemption for a course of a
		studyprogram he intends to follow.
	\item[Preconditions] \
	\begin{itemize}
		\item The student is registrered at the university of the Wellington.
		\item The student wants to follow at least one study contract during the
		upcoming academic year. 
		\item The student has followed at least one course in a previous academic year
		(which can act as a substitute).
		\item The student has chosen the course for which he wants an exemption as part 
		of his course selection.
	\end{itemize}
	\item[Active stakeholders] \ 
	\begin{itemize}
		\item Student: wants to obtain an exemption for a certain course.
	\end{itemize}
	\item[Interested stakeholders] \ 
		\begin{itemize}
		\item Exemption committee: checks the exemption application of the student
		and approves it.
		\end{itemize}
	\item[Normal business flow] \ 
	\begin{enumerate}
	  	% 1
	  	\item The student takes the form to acquire an exemption.
	  	% 2
	  	\item The student fills in his identification information, i.e. the unique
	  	student number on his student card.
	  	% 3
	  	\item The student chooses the course that he/she wants to acquire the
	  	exemption for. 
	  	% 4
	  	\item The student chooses the course that he/she thinks can act as a
	  	substitute for the course in the previous bullet. He/she also provides the
	  	institute where that course was taught.
	  	% 5
	  	\item The student also chooses the study contract to which the exemption
	  	applies.
	  	% 6
	  	\item A motivation why he/she justifies the exemption (e.g. a comparison of
	  	the similarity of the contents of both courses).
	  	% 7
	  	\item The student sends the form to the exemption committee of the faculty
	  	which supervises the studyprogram, \textbf{include} \emph{(UC7: Approve
	  	exemption)}.
	\end{enumerate}
	\item[Alternative business flow] \ 
		\par None
	\item[Exception business flow] \
		\par None
	\item[Outcome (postcondition)] \ 
		\par The student's exemption application is approved.
\end{description}