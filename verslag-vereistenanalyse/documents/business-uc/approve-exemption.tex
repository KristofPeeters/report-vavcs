\section{Approve exemption}

\begin{description}
	\item[Business Event Description] \ 
		\par The exemption committee has received an application for an exemption
		from a student and needs to check if the exemption can be granted.
	\item[Business Use Case Name] \ 
		\par Approve exemption
	\item[Triggering business event] \ 
		\par The exemption committee has received application for an exemption from a
		student.
	\item[Preconditions] \
	\begin{itemize}
		\item The exemption committee has received application for an exemption from a
		student.
	\end{itemize}
	\item[Active stakeholders] \ 
	\begin{itemize}
		\item Exemption committee: checks that the exemption can be granted.
	\end{itemize}
	\item[Interested stakeholders] \ 
		\begin{itemize}
		\item Student: wants to know if his exemption is granted.
		\item Educational Committee: can incorporate any tolerances in approving
		study contracts.
		\end{itemize}
	\item[Normal business flow] \ 
	\begin{enumerate}
	  	% 1
	  	\item The exemption committee glances through the form. 
	  	% 2
	  	\item The exemption committee checks that the specific course is indeed a
	  	valid substitute.
	  	% 3
	  	\item The exemption committee checks that the student fulfills the entry
	  	requirements for the course he wants an exemption for.
	  	% 4
	  	\item The exemption committee approves the exemption.
	  	% 5
	  	\item The exemption committee notifies the student of the approval.
	  	% 6
	  	\item The student is notified.
	\end{enumerate}
	\item[Alternative business flow] \ 
		\par None
	\item[Exception business flow] \ 
	\begin{description}
		\item[6a]  If the exemption committee rejects the exemption then,
		\begin{enumerate}
		  \item The exemption committee notifies the student of the rejection.
		  \item The student is notified.
		\end{enumerate}
	\end{description}
	\item[Outcome (postcondition)] \ 
		\par The student knows about the acceptance of his or her exemption. 
\end{description}
