\section{Apply for exemption}

\begin{description}
	\item[Name] \
		\par Apply for exemption
	\item[Short description] \ 
			\par The student believes that he can get an exemption for a course of a
			program he intends to follow. Such an exemption, however, needs approval from
			a specialized committee of the Wellington university.
	\item[Trigger] \ 
			\par The student wants to acquire an exemption for a certain course.
	\item[Primary actor(s)] \ 
		\begin{itemize}
		  \item Student: wants to acquire an exemption for a certain course.
		\end{itemize}
	\item[Secondary actor(s)] \ 
		\begin{itemize}
		  \item Exemption committee: has to approve the exemption.
		\end{itemize} 
	\item[Preconditions] \ 
	\begin{itemize}
		\item The student is registered at the university of the Wellington.
		\item The student wants to follow at least one study program during the upcoming academic year.
		\item The student has followed at least one course in a previous academic year
		(which can act as a substitute).
		\item The student is logged in.
	\end{itemize}
	\item[Normal flow] \ 
	\begin{enumerate}
	  	% 1
	  	\item The student navigates to the exemption request page.
	  	% 2
	  	\item The system displays the exemption request page with the appropriate
	  	space to specify each piece of the required information.
	  	% 3
	  	\item The student fills in the course that he/she wants to acquire the
	  	exemption for.
	  	% 4
	  	\item The student fills in the course that he/she thinks can act as a
	  	substitute for the course in the previous bullet. He/she also provides the
	  	institute where that course was taught.
	  	% 5
	  	\item The student also fills in the study program to which the exemption
	  	applies.
	  	% 6
	  	\item The student adds a motivation why he/she justifies the exemption (e.g.
	  	a comparison of the similarity of the contents of both courses).
	  	% 7
	  	\item The student submits the request.
	  	% 8
	  	\item The system sends it to the exemption committee of the faculty where
	  	the study program belongs to where it will be approved, \textbf{include}
	  	\emph{(UC5: Approve exemption)}.
	\end{enumerate}
	\item[Alternative flow] \
		\par None
	\item[Postcondition(s)] \ 
	\begin{itemize}
		\item The exemption is approved, both the system and the student are aware of
		this exemption.
	\end{itemize}
	\item[Exception(s)] \ 
	\begin{description}
		\item[1a] If the student already applied for an exemption for the specified
		course, then
		\begin{enumerate}
		  \item The system informs the student that he already applied once.
		  \item The system displays the home screen.
		\end{enumerate}
	\end{description}
\end{description}
