\chapter{Object Constraint Language}

\npar In this section several constraints and business rules are expressed in
OCL. 

\section{Constraints}

\subsection{Given constraints}

\begin{enumerate}
  \item \emph{Entry requirements for course must be free of cycles.} \ 
	\npar \textbf{Context} Course \textbf{inv}:
	\par \hspace*{5 mm} not self.requirements.courses->includes(self)
	
	\item \emph{Students cannot take courses for which they do not satisfy stated
	entry requirements.} \
	\npar \textbf{Context} RegisteredStudent \textbf{inv}:
	\par \hspace*{5 mm} self.courseConfiguration.courses.requirements->
	\par \hspace*{10 mm} \textcolor{Blue}{forAll}(EntryRequirement: req |
	\par \hspace*{15 mm} \textcolor{Blue}{if} req.oclIsKindOf(Strong)
	\textcolor{Blue}{then} \par \hspace*{20 mm}
	self.passedCourses->includesAll(req.courses) 
	\par \hspace*{15 mm} \textcolor{Blue}{else}
	\par \hspace*{20 mm} \textcolor{Blue}{if} req.oclIsKindOf(Weak)
	\textcolor{Blue}{then}
	\par \hspace*{25 mm}
	self.followedCourses->includesAll(req.courses) 
	\par \hspace*{20 mm} \textcolor{Blue}{else}
	\par \hspace*{25 mm} \textcolor{Blue}{if} req.oclIsKindOf(Parallel)
	\textcolor{Blue}{then}
	\par \hspace*{30 mm}
	(self.courseConfiguration.courses->union(self.followedCourses))->
	\par \hspace*{32 mm}includesAll(req.courses)
	\par \hspace*{25 mm} \textcolor{Blue}{else}
	\par \hspace*{30 mm} false
	\par \hspace*{5 mm} )

	\item \emph{Students cannot take courses of some phase if he/she has not taken
	all courses of all preceding phases.}
	
	\npar TODO
	
\end{enumerate}

\subsection{Own constraints}

\begin{enumerate}
	\item \emph{The total amount of study points that a student can take in an
	academic year is in the range from 40 to 75 study points for a full time
	student. For a part-time student this range is from 0 to 30.}
	\npar \textbf{Context} RegisteredStudent \textbf{inv}:
	\par \hspace*{5 mm} \textcolor{Blue}{if} self.oclIsKindOf(FullTimeStudent)
	\textcolor{Blue}{then}
	\par \hspace*{10 mm} self.courseConfiguration.totalStudyPoints $>= 40$ or
	self.courseConfiguration.totalStudyPoints $<= 75$
	\par \hspace*{5 mm} \textcolor{Blue}{else}
	\par \hspace*{10 mm} \textcolor{Blue}{if} self.oclIsKindOf(PartTimeStudent)
	\textcolor{Blue}{then}
	\par \hspace*{10 mm} self.courseConfiguration.totalStudyPoints $>= 0$ or
	self.courseConfiguration.totalStudyPoints $<= 30$
	
	\item \emph{A student can only apply for exam slots of
	courses for which he/she is registered. The same yields for the assigning of
	exam slots.}
	\npar \textbf{Context} RegisteredStudent \textbf{inv}:
	\par \hspace*{5 mm} self.examSlotApplication =
	self.adaptableConfiguration.courses.examSlotApplication
	\par \hspace*{5 mm} and
	\par \hspace*{5 mm} self.examSlotAssignment =
	self.adaptableConfiguration.courses.examSlotAssignment
	
	\npar \textbf{Context} RegisteredStudent \textbf{inv}:
	\par \hspace*{5 mm} self.courseConfiguration.courses->includesAll(self.examSlotApplication)
	\par \hspace*{5 mm} and
	\par \hspace*{5 mm} self.courseConfiguration.courses->includesAll(self.examSlotAssignment)
	
	\item  \emph{A registered student is a student who has a registration and vice
	versa for students who don't have one.}
	\npar \textbf{Context} Student \textbf{inv}:
	\par \hspace*{5 mm} \textcolor{Blue}{if}
	self.oclIsKindOf(RegisteredStudent) \textcolor{Blue}{then}
	\par \hspace*{10 mm} self.registration->size() = 1
	\par \hspace*{5 mm} \textcolor{Blue}{else}
	\par \hspace*{10 mm} self.registration->size() = 0
	
	\item \emph{Students can only select a course configuration which is allowed
	for the study contract they signed.}
	\npar \textbf{Context} RegisteredStudent \textbf{inv}:
	\par \hspace*{5 mm} self.StudyContract->includes(self.adaptableConfiguration)
	
	\item \emph{A grade of a course where the student passed for (PassedResult) is
	always between 0 and 20. The student passes if he obtains a score of 10 or
	more. Or if he tolerates that course.}
	\npar \textbf{Context} PassedResult \textbf{inv}:
	\par TODO
	
	
	\item \emph{The student can tolerate a course with a score of 8 or 9.}
	\npar \textbf{Context} ToleratedResult \textbf{inv}:
	\par \hspace*{5 mm} self.grade = 8 or self.grade = 9
	
	\item \emph{The score of a course that the student followed lies between 0 and
	20.}
	\npar \textbf{Context} FollowedResult \textbf{inv}:
	\par \hspace*{5 mm} self.grade >= 0 and self.grade <= 20
	
	\item \emph{The courses that the student tolerated are the same as the courses
	that his toleration was approved for.}
	\npar \textbf{Context} RegisteredStudent \textbf{inv}:
	\par \hspace*{5 mm} self.tolerated = self.toleranceAssignment
	
\end{enumerate}

\subsection{Use case constraints}

\npar In this section one pre- and one postcondition of the (system) use case
``Register new student'' are modelled in \textsc{ocl}.

\subsubsection{Precondition}

\par \hspace*{5 mm} \textbf{Uses}
\par \hspace*{10 mm} The student willing to register.
\par \hspace*{13 mm} student: Student

\npar \hspace*{5 mm} \emph{The student is not yet registered at Wellington
University.} \par \hspace*{10 mm} student.registration->size() = 0

\subsubsection{Postcondition}

\par \hspace*{5 mm} \emph{The student is now registered at Wellington
University.}
\par \hspace*{10 mm} student.registration->size() = 1
