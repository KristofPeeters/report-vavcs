\chapter{Glossary}
\label{glossary}

\npar In this section there a brief explanation of the concepts used throughout
the report. In case of doubt about the role about an actor, etc. this document
will erase any ambiguity.

\begin{description}
\item[]
\item[StudyContract] is one of the three types of contracts (diploma, credit or
exam) but is purely the contract itself. This study contract contains no
information on any courses, examslots, tolerances or exemptions.
\item[StudyProgram] is the whole of all contracts of one single student
augmented with all information on the courses (i.e. exam slots, exemptions and
tolerances) followed in all contracts.
\item[Student] are persons who are able to register at the university of
Wellington. They differ with a ``Person'' in the sense that children are
not encorporated. Of course highly intelligent children can of course fall under
the scope of the concept Student because all they need is a high school degree
to be able to register.
\item[RegisteredStudent] are registered students at the wellington university.
The possess a student card and corresponding login credentials.
\item[User] is a person, known to the Wellington university, who has rights to
log in (and of course log off) to online study board to perform one or more
tasks.
\item[Exemption committee] This council consists of all the people of the
Wellington university who are in charge of (dis)approving exemptions for
courses.
\item[Registration committee] is a council consisting of all people who have
rights to register (new) students at the Wellington university.
\item[Educational committee] consists of all people who are responsible for
approving study program
\end{description}
